\documentclass[conference,compsoc]{IEEEtran}

% *** CITATION PACKAGES ***
%
\ifCLASSOPTIONcompsoc
  % IEEE Computer Society needs nocompress option
  % requires cite.sty v4.0 or later (November 2003)
  \usepackage[nocompress]{cite}
\else
  % normal IEEE
  \usepackage{cite}
\fi

\usepackage{graphicx}
\graphicspath{ {images/} }
\usepackage{listings}
\usepackage[final]{pdfpages}

% correct bad hyphenation here
\hyphenation{op-tical net-works semi-conduc-tor}


\begin{document}
\begin{titlepage}
	\centering
	\includegraphics[width=0.5\textwidth]{McGill_Logo}\par\vspace{1cm}
	{\scshape\LARGE McGill University \par}
	\vspace{1cm}
	{\scshape\Large ECSE 415 Project\par}
	\vspace{1.5cm}
	{\huge\bfseries Cat and dog\par}
	\vspace{2cm}
	{\Large Jaeho Lee\par}
	{ 260633759\par} {\Large Raymond So \par} {260640297 \par}
	\vfill
	submitting to\par
	Prof.~James \textsc{Clark}

	\vfill

% Bottom of the page
	{\large \today\par}
\end{titlepage}
%
% paper title
% Titles are generally capitalized except for words such as a, an, and, as,
% at, but, by, for, in, nor, of, on, or, the, to and up, which are usually
% not capitalized unless they are the first or last word of the title.
% Linebreaks \\ can be used within to get better formatting as desired.
% Do not put math or special symbols in the title.
\title{ECSE 415 Project\\ Cat and Dog}
\author{\IEEEauthorblockN{Jaeho Lee \IEEEauthorrefmark{1} and Raymond So \IEEEauthorrefmark{2}}
\IEEEauthorblockA{Department of Electrical Engineering, McGill University\\
\IEEEauthorrefmark{1}jaeho.lee2@mail.mcgill.ca, \IEEEauthorrefmark{2}raymond.su@mail.mcgill.ca}}

\maketitle

% As a general rule, do not put math, special symbols or citations
% in the abstract
\begin{abstract}
A two stage CMOS amplifier circuit is designed in this project. The amplifier consumes \(67.0\mu W\) when operating. The design of amplifier were modified compare to design from previous phase. The voltage and current outputs were closed to prediction. However the gain had \(\approx 1.8 dB\) difference which is \(23.0 \%\). Which is reasonable consider numerous approximations and assumptions were made for the calculation.
\end{abstract}

\IEEEpeerreviewmaketitle

\section{Introduction}
% no \IEEEPARstart
In this project, a two stage CMOS amplifier circuit is designed to be able to used in a unity gain closed loop configuration to act as a buffer between a tire sensor and an ADC in microcontroller in an automobile. The amplifier is operated in specific voltage depends on students' ID number. The amplifier is consisted of two current mirrors, a differential amplifier and a cascade circuit. In this project, by closing loop, unity gain circuit is tested and designed based on settling time, steady-state error. The report contains how the open-loop amplifier gain and bandwidth is related to the closed-loop step response parameter, how this design is related to the frequency compensation process, whether previously designed open loop amplifier's parameters-from phase 1 report-were used for the project and comparison between the numerical calculation and SPICE simulation result.

\subsection{Amplifier requirements}
The student id of the student is 260633759. The supply voltage level can be calculated based on the equation:\\
\[
V_{ss} = 5 + \frac{4^{th} + 3^{rd} \textit{ digits from the end} }{20}
\]
the minimum gain \(A_o\) is:\\
\[
A_c = \frac{1}{1 + \frac{1}{•A_o}}
\]
Since settling time is 100 \(\mu \)s, and \(\textit{settling time} \approx 5\tau\) where the equation for bandwidth is:\\
\[
\frac{1}{2\pi\tau}
\]
From the project description and after all calculation were made, a table for requirements is follow:\\
\begin{table}[!hbt]
	\centering
	\begin{tabular}{|c|c|}
	\hline
	DC Output Supply Voltage (V) & \(\pm 6.85V\)\\
	Settling time (s) & \(< 100 \mu \)s \\
	Input Step (V) & \(10 mV\)\\
	Overshoot Voltage (V) & \(\leq 10.5 mV\)\\
	Steady-State Error (V) & \(<10 \mu V\)\\
	Transistors & MOSFET CD4007\\
	Load Capacitor & \(5 pF\)\\
	Amplifier Gain & \(A_o \geq\ 999, 60 dB\)\\
	Closed-loop Bandwidth & \(\geq 7.96 kHz\)\\
	\hline
	\end{tabular}
	\caption{Amplifier Requirements}\label{table: 1}
\end{table}
\section{Discussion}
\subsection{Circuit description}
The CMOS two-stage amplifier consists of two current
mirrors, a differential amplifier and a cascade circuit. The circuit diagram is drawn in figure 1. For ease of calculation, the circuit is designed to have 1V \(V_{ov}\) for all MOSFETs.  A value of \(50 k\Omega\) was chosen for reference resistor, \(R\_ref\) and \(0.1 pF\) was chosen for compensation capacitor, \(C\_c\). Using Equation (1) , this design is calculated to generate an open-loop gain, \(A_o\), of 1254 which satisfies the requirements. Using the circuit, simulated result clearly does not satisfies the design requirement. The output voltage, \(V\_o\), has a massive continuous pulse which does not satisfy the design requirement which addresses that both \(R\_ref\) and \(C\_c\) values have to be changed. This is depicted in figure 2.

In industry, normally \(100\Omega\) to \(50M\Omega\) resistors and \(0.1pF\) to \(10\mu F\) capacitors are used for circuit design. Consider these values are design restriction, \(15M\Omega\) resistor \(25pF\) capacitor are chosen for \(R\_ref\) and \(C\_c\). Since very high resistance value was chosen for reference resistor, it is obvious that current goes through second layer amplifier, cascade amplifier, would be low. In order to fix this issue, 2 PMOSs and 2 NMOSs are added for cascade amplifier to triple MOSFET width. Figure 3 presents the circuit diagram. A detail figure is provided in appendices\\


\subsection{Open-Loop Analysis}

To simplify calculations, the amplifier is designed to give identical \(V_{ov}\) for all MOSFETs. Equation (1) was used to calculate \(A_o\).
\begin{equation}
A_o = \frac{2*V_{an} * V_{ap}}{V_{ov}*(V_{an}+V_{ap})}^2
\end{equation}\\
The gain in decibel is simply calculated by:
\[
A_{dB} = 20 \log_{10}(A_o)
\]
A Microsoft Excel spreadsheet was created for the calculation. The comparison between calculated values and SPICE simulated values are as follows:\\
\begin{table}[!htb]
	\begin{center}
	\begin{tabular}{|c|c|c|}
	\hline
	&Calculation & Simulation\\
	\hline	
	\(V_{d1}\) & -4.782 V&-4.782 V\\
	\(V_{d2}\) & -2.068 V&-2.063 V\\
	\(V_{d3}\) & 5.282&5.286\\
	\(i_1\)&773.21\(n\)A&775.49\(n\)A\\
	\(i_2\)&1.546\(\mu\)A&1.551\(\mu\)A\\
	\(i_4\)&776.70\(n\)A&775.49\(n\)A\\
	\(i_5\)&769.70\(n\)A&796.11\(n\)A\\
	\(i_6\)&776.70\(n\)A&796.11\(n\)A\\
	\(i_8\)&776.70\(n\)A&840.66\(n\)A\\
	\(V_{ov}\)&68.3\(mV\)&69.15\(mV\)\\
	\(A_{dB}\)&110.70 dB &112.54 dB\\
	\hline
	\end{tabular}
	\end{center}
	\caption{Calculation \& Simulation Comparison}\label{table2 :}
\end{table}\\
The graph with gain in decibel and phase margin at 0dB is provided in figure 4.\\
%\begin{figure}[!htb]
%\centering
%\includegraphics[width=\columnwidth]{2openloop}
%\caption{Open-loop Gain}
%\end{figure}\\

The graph addresses that based on equation (4), the phase margin is positive which states the circuit is stable. The table above indicates that the predicted voltage values and current values made by the student are fairly close to SPICE simulation values. The calculated gain based on the equation (1) is 110.49 dB where the simulated gain is 112.54 dB.\\
\\The entire amplifier is running in \(2 * 6.85V = 13.7V\). The current goes through voltage source, \(V_{dd}\&V_{ss}\) is -4.8900\(\mu\)A. Therefore the power consumption of the amplifier is 13.7V *-4.8900\(\mu\)A =67.0\(\mu\)W

\subsection{Closed-Loop Analysis}
In order to start closed loop analysis, few equations must be discussed. Poles are critical components in order to stabilize the circuit. 4 poles are observed in this amplifier design where two poles are on the same spot. Open-loop simulation result is provided in figure 4, Closed-loop is provided in figure  (5). Poles define bandwidth of the circuit. Required closed-loop bandwidth for the project is \(\tau \geq 7.96kHz\). Where \(\tau\) can be represented as follow equation (2) and the bandwidth is addressed on equation (3). Phase margin calculation is on equation (4).
\begin{equation}
\tau = \sum_{n=1}^{N} C_nR_n
\end{equation}\\
\begin{equation}
\omega = \frac{1}{2\pi\tau}
\end{equation}\\
\begin{equation}
PM = f_{0} + 180^{\circ}
\end{equation}\\

In both Closed-Loop and Open-Loop simulation, the phase margin is positive which addresses the circuit is stable. To predict where would each pole appears, following equations were used.
\begin{equation}
\omega_1 = \frac{1}{R_o*C_c}
\end{equation}
\begin{equation}
\omega_2 = \frac{1}{R_{o2}*C_L}
\end{equation}
\begin{equation}
\omega_3 = \frac{1}{2*R_{o6,7}*C_{gs2}}
\end{equation}\\
These equations were used to guide the estimation so we could predict around where the pole would exist with different values. However since the difference between prediction and simulation was fairly high, various simulations were made to adjust amplifier in order to satisfy the requirements. Based on figure (5), the bandwidth of the amplifier is \(389.02kHz\) which satisfies the design requirements from table (1)\\

With the fully built Closed-Loop amplifier circuit, the simulation results to verify whether the design satisfies requirements are following on figure (6), (7).


Both figures address that the amplifier satisfies all requirements, peak overshoot, settling time and steady-state error.  Based on the results, table 3 was made to conclude that the design successfully satisfy all the requirements.\\
\begin{table}[!hbt]
	\centering
	\begin{tabular}{|c|c|c|}
	\hline
	&Requirements&Result\\
	\hline	
	Settling time (s) & \(< 100 \mu \)s & \(\approx 30\mu \)s \\
	Input Step (V) & \(10 mV\)&\(9.9999mV\)\\
	Overshoot Voltage (V) & \(\leq 10.5 mV\)&\(10.389 mV\)\\
	Steady-State Error (V) & \(<10 \mu V\)&\(8.19\mu V\)\\
	Amplifier Gain & \(A_o \geq\ 999, 60 dB\)&\(112.5dB\)\\
	Closed-loop Bandwidth & \(\geq 7.96 kHz\)&\(389.02kHz\)\\
	\hline
	\end{tabular}
	\caption{Requirements and Result}\label{table: 3}
\end{table}\\
\section{Conclusion}

\begin{thebibliography}{1}

\bibitem{IEEEhowto:kopka}
K~. Smith, A~. Smith, \emph{Microelectronic circuits(Oxford series in electrical and computer engineering (Hardco)}, 7th~ed. New York, NY, United States: Oxford University Press, 2014.

\end{thebibliography}

\onecolumn

\section{Appendices}
\noindent SPICE Netlist:\\
\begin{lstlisting}


\end{lstlisting}


%\includepdf[landscape=true]{1.pdf}


% trigger a \newpage just before the given reference
% number - used to balance the columns on the last page
% adjust value as needed - may need to be readjusted if
% the document is modified later
%\IEEEtriggeratref{8}
% The "triggered" command can be changed if desired:
%\IEEEtriggercmd{\enlargethispage{-5in}}

% references section

% can use a bibliography generated by BibTeX as a .bbl file
% BibTeX documentation can be easily obtained at:
% http://mirror.ctan.org/biblio/bibtex/contrib/doc/
% The IEEEtran BibTeX style support page is at:
% http://www.michaelshell.org/tex/ieeetran/bibtex/
%\bibliographystyle{IEEEtran}
% argument is your BibTeX string definitions and bibliography database(s)
%\bibliography{IEEEabrv,../bib/paper}
%
% <OR> manually copy in the resultant .bbl file
% set second argument of \begin to the number of references
% (used to reserve space for the reference number labels box)




% that's all folks
\end{document}


